% Options for packages loaded elsewhere
\PassOptionsToPackage{unicode}{hyperref}
\PassOptionsToPackage{hyphens}{url}
%
\documentclass[
]{book}
\title{Lecture Note on Terwilliger Algebra}
\author{P. Terwilliger, edited by H. Suzuki}
\date{2022-11-10}

\usepackage{amsmath,amssymb}
\usepackage{lmodern}
\usepackage{iftex}
\ifPDFTeX
  \usepackage[T1]{fontenc}
  \usepackage[utf8]{inputenc}
  \usepackage{textcomp} % provide euro and other symbols
\else % if luatex or xetex
  \usepackage{unicode-math}
  \defaultfontfeatures{Scale=MatchLowercase}
  \defaultfontfeatures[\rmfamily]{Ligatures=TeX,Scale=1}
\fi
% Use upquote if available, for straight quotes in verbatim environments
\IfFileExists{upquote.sty}{\usepackage{upquote}}{}
\IfFileExists{microtype.sty}{% use microtype if available
  \usepackage[]{microtype}
  \UseMicrotypeSet[protrusion]{basicmath} % disable protrusion for tt fonts
}{}
\makeatletter
\@ifundefined{KOMAClassName}{% if non-KOMA class
  \IfFileExists{parskip.sty}{%
    \usepackage{parskip}
  }{% else
    \setlength{\parindent}{0pt}
    \setlength{\parskip}{6pt plus 2pt minus 1pt}}
}{% if KOMA class
  \KOMAoptions{parskip=half}}
\makeatother
\usepackage{xcolor}
\IfFileExists{xurl.sty}{\usepackage{xurl}}{} % add URL line breaks if available
\IfFileExists{bookmark.sty}{\usepackage{bookmark}}{\usepackage{hyperref}}
\hypersetup{
  pdftitle={Lecture Note on Terwilliger Algebra},
  pdfauthor={P. Terwilliger, edited by H. Suzuki},
  hidelinks,
  pdfcreator={LaTeX via pandoc}}
\urlstyle{same} % disable monospaced font for URLs
\usepackage{longtable,booktabs,array}
\usepackage{calc} % for calculating minipage widths
% Correct order of tables after \paragraph or \subparagraph
\usepackage{etoolbox}
\makeatletter
\patchcmd\longtable{\par}{\if@noskipsec\mbox{}\fi\par}{}{}
\makeatother
% Allow footnotes in longtable head/foot
\IfFileExists{footnotehyper.sty}{\usepackage{footnotehyper}}{\usepackage{footnote}}
\makesavenoteenv{longtable}
\usepackage{graphicx}
\makeatletter
\def\maxwidth{\ifdim\Gin@nat@width>\linewidth\linewidth\else\Gin@nat@width\fi}
\def\maxheight{\ifdim\Gin@nat@height>\textheight\textheight\else\Gin@nat@height\fi}
\makeatother
% Scale images if necessary, so that they will not overflow the page
% margins by default, and it is still possible to overwrite the defaults
% using explicit options in \includegraphics[width, height, ...]{}
\setkeys{Gin}{width=\maxwidth,height=\maxheight,keepaspectratio}
% Set default figure placement to htbp
\makeatletter
\def\fps@figure{htbp}
\makeatother
\setlength{\emergencystretch}{3em} % prevent overfull lines
\providecommand{\tightlist}{%
  \setlength{\itemsep}{0pt}\setlength{\parskip}{0pt}}
\setcounter{secnumdepth}{5}
\usepackage{booktabs}
\usepackage{amsthm}
\makeatletter
\def\thm@space@setup{%
  \thm@preskip=8pt plus 2pt minus 4pt
  \thm@postskip=\thm@preskip
}
\makeatother
\ifLuaTeX
  \usepackage{selnolig}  % disable illegal ligatures
\fi
\usepackage[]{natbib}
\bibliographystyle{apalike}

\usepackage{amsthm}
\newtheorem{theorem}{Theorem}[chapter]
\newtheorem{lemma}{Lemma}[chapter]
\newtheorem{corollary}{Corollary}[chapter]
\newtheorem{proposition}{Proposition}[chapter]
\newtheorem{conjecture}{Conjecture}[chapter]
\theoremstyle{definition}
\newtheorem{definition}{Definition}[chapter]
\theoremstyle{definition}
\newtheorem{example}{Example}[chapter]
\theoremstyle{definition}
\newtheorem{exercise}{Exercise}[chapter]
\theoremstyle{definition}
\newtheorem{hypothesis}{Hypothesis}[chapter]
\theoremstyle{remark}
\newtheorem*{remark}{Remark}
\newtheorem*{solution}{Solution}
\begin{document}
\maketitle

{
\setcounter{tocdepth}{1}
\tableofcontents
}
\hypertarget{about-this-lecturenote}{%
\chapter*{About this lecturenote}\label{about-this-lecturenote}}
\addcontentsline{toc}{chapter}{About this lecturenote}

\hypertarget{setting}{%
\section*{Setting}\label{setting}}
\addcontentsline{toc}{section}{Setting}

This note is created by \texttt{bookdown} package on RStudio.

\begin{enumerate}
\def\labelenumi{\arabic{enumi}.}
\tightlist
\item
  Log-in to my GitHub Account
\item
  Go to RStudio/bookdown-demo repository: \url{https://github.com/rstudio/bookdown-demo}
\item
  Use This Template
\item
  Input Repository Name
\item
  Select Public - default
\item
  Create repository from template
\item
  From Code download ZIP
\item
  Move the extracted folder into a favorite directory
\item
  Open RStudio Project in the folder
\item
  Use Terminal in the buttom left pane

  \begin{itemize}
  \tightlist
  \item
    confirm that the current directory is the home directry of the project by pwd
  \end{itemize}
\item
  (failed to proceed by ssh)
\item
  Use Console

  \begin{enumerate}
  \def\labelenumii{\arabic{enumii}.}
  \tightlist
  \item
    library(usethis)
  \item
    use\_git()
  \item
    use\_github() --- Error
  \item
    gh\_token\_help()
  \item
    create\_github\_token(): create a token in the github page. Copy the token
  \item
    gitcreds::gitcreds\_set(): paste the token, the token is to be expired in 30 days
  \end{enumerate}
\item
  Use Terminal

  \begin{enumerate}
  \def\labelenumii{\arabic{enumii}.}
  \tightlist
  \item
    git remote add origin \url{https://github.com/icu-hsuzuki/t-alagebra.git}
  \item
    git push -u origin main
  \item
    type in the password of the computer
  \end{enumerate}
\item
  Use GIT in R Studio
\end{enumerate}

\hypertarget{access-from-another-host}{%
\section*{Access from Another Host}\label{access-from-another-host}}
\addcontentsline{toc}{section}{Access from Another Host}

\begin{enumerate}
\def\labelenumi{\arabic{enumi}.}
\tightlist
\item
  library(usethis)
\item
  use\_git()
\item
  create\_github\_token()
\item
  gitcreds::gitcreds\_set(): Replace these credentials
\end{enumerate}

\hypertarget{lec1}{%
\chapter{Lecture 1}\label{lec1}}

\hfill [Wednesday, January 20, 1993]{style="float:right"}

A graph (undirected, without loops or multiple edges) is a pair \(\Gamma = (X, E)\), where
\begin{align}
X &= \textrm{finite set (of vertices)}\\
E & = \textrm{set of (distinct) 2-element subsets of }X \textrm{ (= edges of ) }\Gamma.
\end{align}
vertices \(x\) and \(y\in X\) are adjacent if and only if \(xy\in E\).

\begin{example}
Let \(\Gamma\) be a graph. \(X = \{a, b, c, d\}\), \(E = \{ab, ac, bc, bd\}\).
\end{example}

Set \(n = |X|\), the order of \(\Gamma\).

Pick a field \(K\) (\(=\mathbb{R}\) or \(\mathbb{C}\)). Then \(\mathrm{Mat}_X(K)\) denotes the \(K\) algebra of all \(n\times n\) matrices with entries in \(K\). (rows and columns are indexed by \(X\))

Adjacency matrix \(A\in \mathrm{Mat}_X(K)\) is defined by
\begin{align}
A_{xy} & = \left\{\begin{array}{cl} 1 & \textrm{ if } \; xy\in E\\
0 & \textrm{ else } . \end{array}\right.
\end{align}

\hypertarget{lec2}{%
\chapter{Lecture 2}\label{lec2}}

Here is a review of existing methods.

\hypertarget{lec3}{%
\chapter{Lecture 3}\label{lec3}}

We describe our methods in this chapter.

Math can be added in body using usual syntax like this

\hypertarget{math-example}{%
\section{math example}\label{math-example}}

\(p\) is unknown but expected to be around 1/3. Standard error will be approximated

\[
SE = \sqrt(\frac{p(1-p)}{n}) \approx \sqrt{\frac{1/3 (1 - 1/3)} {300}} = 0.027
\]

You can also use math in footnotes like this\footnote{where we mention \(p = \frac{a}{b}\)}.

We will approximate standard error to 0.027\footnote{\(p\) is unknown but expected to be around 1/3. Standard error will be approximated

  \[
  SE = \sqrt(\frac{p(1-p)}{n}) \approx \sqrt{\frac{1/3 (1 - 1/3)} {300}} = 0.027
  \]}

\hypertarget{applications}{%
\chapter{Applications}\label{applications}}

Some \emph{significant} applications are demonstrated in this chapter.

\hypertarget{example-one}{%
\section{Example one}\label{example-one}}

\hypertarget{example-two}{%
\section{Example two}\label{example-two}}

\hypertarget{final-words}{%
\chapter{Final Words}\label{final-words}}

We have finished a nice book.

  \bibliography{book.bib,packages.bib}

\end{document}
